\externaldocument{Introduction.tex}
\externaldocument{Experimental_Apparatus.tex}
\externaldocument{Results.tex}
\externaldocument{Analysis.tex}
\externaldocument{Checks_and_Systematics.tex}
\externaldocument{EIC_Jets.tex}
\externaldocument{Discussion.tex}

\chapter{Conclusions}

The $\gammaiso$-hadron correlations presented in this work marks the first \gammaiso-tagged fragmentation function measurement in \pPb. The relatively lower $Q^2$ probed include partons that are expected to be more sensitive to cold nuclear matter modification. Nonetheless, no modification is observed within the reported uncertainties between the fragmentation function in \pPb~and pp. Comparisons to models indicate that any cold nuclear matter effects, or suppression from a small droplet of QGP formed in \pPb~should be small, and fall within the reported uncertainties of this measurement. 

Any modification observed in Pb--Pb larger than the uncertainty of 13\% cannot be attributed to cold nuclear matter effects, and must be due to the modification of the produced Quark Gluon Plasma on the scattered parton. An analysis on isolated photon-hadron correlations in Pb--Pb using similar techniques is a natural extension of this work, and would focus on hot nuclear matter effects. For additional insights into cold nuclear matter effects, and their modification of scattered partons, however, a very different direction is needed. While the golden channel of prompt $\gamma$-jet and $\gamma$-hadron correlations provide excellent constraints on the kinematics of the scattered parton, and therefore remarkable insight into modifications of the parton, they remain relatively rare in high energy heavy ion collisions due to the relatively low cross section of prompt photons compared to decay photons. Thus the overall statistical precision of the measurement is limited. 

The Electron Ion Collider encompasses another direction for the study of cold nuclear mater. Semi Inclusive Deep Inelastic Scattering where the final state jet $Q^2$ can be precisely measured is a promising direction. The EIC with an integrated luminosity of $\mathcal{L}_{EIC}=10fb^{-1}$, will enable measurements with much higher statistical precision, and more rigorous comparisons to theoretical predictions. Specifically, measurements of charged jets represent two aspects of the larger jet physics program at the EIC. The charged jet fragmentation function can be measured at much lower $Q^2$ than in relativistic heavy ion or p+A collisions. The electron-jet correlation functions can probe the interaction of scattered partons with a cold nuclear medium. 

The study of cold nuclear matter effects at ALICE can help us elucidate differences hot nuclear matter effects arising in a small quark gluon plasma droplet vs. cold nuclear matter effects. These can be compared to transport in QCD matter at the highest energy densities available. Studies at the EIC will more directly probe QCD in nuclei, and also help answer fundamental questions relating the internal structure of nuclei. As these measurements progress in precision and sensitivity, our knowledge of QCD inside the nucleus will grow, and with it, our understanding of all hadronic matter that makes up our visible universe.

