\chapter{Conclusions}

The $\gammaiso$-hadron correlations presented in this work marks the first \gammaiso-tagged fragmentation function measurement in \pPb. The relatively lower $Q^2$ probed include partons that are expected to more sensitive to cold nuclear matter modification. Nonetheless, no modification is observed within the reported uncertainties between the fragmentation function in \pPb and pp. Additionally, comparisons to models indicate that any cold nuclear matter effects, or the suppression from a small droplet of QGP formed in \pPb are expected to be small, and also fall well within the reported uncertainties. This means that any modification observed in Pb--Pb that is larger than the integrated uncertainty of 13\% cannot be attributed to cold nuclear matter effects, and must be due to the modification of the produced Quark Gluon Plasma on the scattered parton.

An analysis of isolated photon-hadron correlations in Pb--Pb using the similar techniques is a natural extension of this work, and a measurement that would focus on hot nuclear matter effects. For additional insights into cold nuclear matter effects, and their modification to scattered partons, however, a very different direction is needed. While the golden channel of prompt $\gamma$-jet and $\gamma$-hadron correlations provide excellent constraints on the kinematics of the scattered parton, and therefore remarkable insight into modifications of the parton, they remain relatively rare measurable event in high energy heavy ion collisions due to the relatively low cross section of prompt photons compared to decay photons, and thus the overall statistical precision of the measurement is limited. The Electron Ion Collider encompasses another direction. Semi Inclusive Deep Inelastic Scattering where the final state jet the $Q^2$ of the collision can be precisely measured at the EIC with an integrated luminosity of $\mathcal{L}_{EIC}=10fb^{-1}$. This not only enables measurements with much higher statistical precision, but also more rigorous comparisons to theoretical models. Specifically, measurements charged jets simulated in this represent just two aspects of the larger jet physics program at the EIC. The charged jet fragmentation function can be measured at much lower $Q^2$ than in relativistic heavy ion collisions, while the electron-jet correlation functions can probe the interaction of scattered partons through a cold nucleus as a medium. 

The study of cold nuclear matter effects at ALICE can help us elucidate differences hot nuclear matter effects arising from the formation of the quark gluen plasma vs. cold nuclear matter effects attributed simply to the presence of a nucleus in the collision, all while probing the QCD phase diagram at the highest energy densities available. Studies at the EIC can more directly probe the behaviour of QCD in nuclei, and also help answer fundamental questions relating the internal structure of nuclei. As these measurements progress in precision and sensitivity, our knowledge of QCD inside the nucleus grows, and with it, our understanding of all hadronic matter that makes up our visible universe.

